\documentclass[12pt,a4paper]{report}
\pagenumbering{gobble}

\begin{document}
		\begin{center}
			\vspace{0.5cm}
			\Large \textsc{Abstract}
		\end{center}
	\hfill \textsc{Alexandru Bajzat}
	\vspace{1.5cm}
	
	
	
	This paper looks at how machine learning has influenced the image processing domain and it presents approaches and algorithms that have the best usability upon it. The goal is to find a method of constructing intelligent systems which best scale on mobile devices, allowing them to complete the inference computation of the machine learning phases. In order to achieve this goal, two separated platforms which handle different steps of the learning and prediction job need to be created. Each step in the development of the platforms and their correlation is strongly supported by theoretical and practical explanations. The originality is given by the all-in-one model training platform and mobile device compatible model export.
	
	This paper presents mobile device`s evolution towards artificial intelligence and how it coped with complexity increase in machine learning`s algorithms. 
	The paper briefly introduces system architectures and describes differences that each bring in the way they manage computation and energy consumption.
	
	The first chapter gives a brief description about machine learning importance followed by it`s applicability and presents the structure of the paper.
	
	The second chapter dives into a more theoretical approach of machine learning. It gives a short,how and why it started, history of the artificial intelligence branch. 
	
	The third chapter presents how mobile phones evolved into the artificial intelligence computation capable devices they are today. It presents the history of mobile phones, from the very beginning, and their system processors progress. The paper introduces two big architectures, Intel x86 and ARM, in order to present differences between two major computational approaches.
	
	The forth chapter presents the development step of the model training platform and the mobile phone application as well as all the features between them.
	
	The last chapter comes with a conclusion about the constraints, results and possible improvements.
	
	This work is the result of my own activity. I have neither given nor received unauthorized assistance on
	this work.
\end{document}