\chapter{Introducere}

Invăţarea automată este un concept ce se regaseşte tot mai mult in viaţa noastră de zi cu zi prin intermediul dispozitivelor din jurul nostru. Recunoasterea vizuală, fiind o implementare a invăţării automate, este un proces foarte costisitor din punct de vedere al memoriei si al forţei de computaţie, datorită avansării în tehnologie pe planul imagistic. Această ramura a invăţării automate se regăseşte intr-o gama largă de aplicaţii, având diferite aplicabilităţi.  \newline

Scopul acestui proiect este de a integra un sistem inteligent de recunoastere a notelor muzicale pe portativ, utilizând reţele neuronale convoluţionale, pe sistemul unui dispozitiv mobile cu sistem de operare Android. Lucrarea va urmări crearea unei platforme capabile de a configura şi antrena un model, astfel incât să putem obţine o compatibilitate optimă pentru puterea de procesarea a telefonului mobil.\newline

Proiectul a fost realizat utilizând o platformă de antrenare a reţelelor neuronale convoluţionale, rulată pe un server si distribuită prin TCP/IP clienţilor, i.e dispozitive mobile, unde sunt integrate printr-o interfaţă de configurare si inferenţă. Setul de date asupra căruia s-a realizat antrenarea, este realizat manual utilizând camera unui dispozitiv mobil, trecut printr-un proces de augmentare. 
Aplicaţia mobilă prezintă o interfaţă grafică minimală, în scopul punerii la dispoziţie a capabilităţii modelului antrenat si rulat local pe dispozitiv.\newline

Primul capitol al lucrării prezintă in ansamblu invăţarea automată si subramura sa, procesarea de imagini. Acesta prezintă succint etape istorice in dezvoltarea invăţării automate, evidenţeaza conceptele de baza şi introduce procedee specializate in procesarea de imagini.
Cel de al doilea capitol al lucrării prezintă linia de progres al dispozitivelor mobile, precum şi specificaţii hardware si metode de accelerare a computaţiei destinată operaţiilor realizate de agoritmii inteligenţi. Sunt prezentate pe scurt arhitecturi ale procesoarelor utilizate până în prezent de către dispozitivele mobile şi sistemul de operare Android.
Cel de al treilea capitol urmăreste etapa de proiectare şi implementare, atât al platformei pentru antrenarea modelului cat şi a aplicaţiei mobile destinată utilizării modelului rezultat.