\chapter{Introducere}

Învăţarea automată este un concept ce se regăseşte tot mai mult în viaţa noastră de zi cu zi prin intermediul dispozitivelor din jurul nostru. Recunoașterea vizuală, fiind o implementare a învăţării automate, este un proces foarte costisitor din punct de vedere al memoriei și al forţei de computaţie, datorită avansării în tehnologie pe planul imagistic. Această ramura a învăţării automate se regăseşte într-o gamă largă de aplicaţii, având diferite aplicabilităţi.  \newline

Scopul acestui proiect este de a integra un sistem inteligent de recunoaștere a notelor muzicale pe portativ, utilizând reţele neuronale convoluţionale, pe sistemul unui dispozitiv mobil cu sistem de operare Android. Lucrarea va urmări crearea unei platforme capabile de a configura şi antrena un model, astfel încât să putem obţine o compatibilitate optimă pentru puterea de procesarea a telefonului mobil.\newline

Proiectul a fost realizat utilizând o platformă de antrenare a reţelelor neuronale convoluţionale, rulată pe un server si distribuită prin TCP/IP clienţilor, i.e dispozitive mobile, unde sunt integrate printr-o interfaţă de configurare și inferenţă. Setul de date asupra căruia s-a realizat antrenarea, este realizat manual utilizând camera unui dispozitiv mobil, trecut printr-un proces de augmentare. 
Aplicaţia mobilă prezintă o interfaţă grafică minimală, în scopul punerii la dispoziţie a capabilităţii modelului antrenat și rulat local pe dispozitiv.\newline

Primul capitol al lucrării prezintă în ansamblu învăţarea automată și subramura sa, procesarea de imagini. Acesta prezintă succint etape istorice în dezvoltarea învăţării automate, evidenţeaza conceptele de bază şi introduce procedee specializate în procesarea de imagini.
Cel de al doilea capitol al lucrării prezintă linia de progres al dispozitivelor mobile, precum şi specificaţii hardware și metode de accelerare a computaţiei destinată operaţiilor realizate de agoritmii inteligenţi. Sunt prezentate pe scurt arhitecturi ale procesoarelor utilizate până în prezent de către dispozitivele mobile şi sistemul de operare Android.
Cel de al treilea capitol urmăreste etapa de proiectare şi implementare, atât al platformei pentru antrenarea modelului cât şi a aplicaţiei mobile destinată utilizării modelului rezultat.