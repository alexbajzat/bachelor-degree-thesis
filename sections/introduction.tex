\chapter{Introducere}

Invatarea automata este un concept ce se regaseste tot mai mult in viata noastra de zi cu zi prin intermediul dispozitivelor din jurul nostru. Recunoasterea vizuala, fiind o implementare a invatarii automate, este un proces foarte costisitor din punct de vedere al memoriei si al fortei de computatie, datorita avansarii in tehnologie pe planul imagistic. Aceasta ramura a invatarii automate se regaseste intr-o gama larga de aplicatii, avand diferite aplicabilitati.  \newline

Scopul acestui proiect este de a integra un sistem inteligent de recunoastere a notelor muzicale pe portativ, utilizand retele neuronale convolutionale, pe sistemul unui dispozitiv mobile cu sistem de operare Android. Lucrarea va urmari crearea unei platforme capabile de a configura si antrena un model, astfel incat sa putem obtine o compatibilitate optima pentru puterea de procesarea a telefonului mobil.\newline

Proiectul a fost realizat utilizand o platforma de antrenare a retelelor neuronale convolutionale, rulata pe un server si distribuita prin TCP/IP clientilor, i.e dispozitive mobile, unde sunt integrate printr-o interfata de configurare si inferenta. Setul de date asupra caruia s-a realizat antrenarea, este realizat manual utilizand camera unui dispozitiv mobil, trecute printr-un proces de augmentare. 
Aplicatia mobila prezinta o interfata grafica minimala, in scopul punerii la dispozitie a capabilitatii modelului antrenat si rulat local pe dispozitiv.\newline

Primul capitol al lucrarii prezinta in ansamblu invatarea automata si subramura a sa, procesarea de imagini. Acesta prezinta succint etape istorice in dezvoltarea invatarii automate, evidenteaza conceptele de baza si introduce procedee specializate in procesarea de imagini.
Cel de al doilea capitol al lucrarii prezinta linia de progres al dispozitivelor mobile, precum si specificatii hardware si metode de accelerare a computatie destinata operatiilor realizate de agoritmii inteligenti. Sunt prezentate pe scurt arhitecturii ale procesoarelor utilizate pana in prezent de catre dispozitivele mobile si sistemul de operare Android.
Cel de al treilea capitol urmareste etapa de proiectare si implementare, atat al platformei pentru antrenarea modelului cat si a aplicatiei mobile destinata utilizarii modelului rezultat.