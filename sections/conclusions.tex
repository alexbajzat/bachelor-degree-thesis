\chapter{Concluzii}

Domeniul procesarii de imagini este unul a carui desfasurare este in amploare si care va fi nelipsit din majoritatea aplicatiilor. Datorita avansarii rapide a tehnologiei si a dependentelor costisitoare ce isi pun amprenta asupra algoritmilor de invatare automata, domeniul inteligent va trebui sa tina pasul atat din punctul de vedere tehnic cat si din punctul de vedere al algoritmilor si a optimizarii. \newline

In aceasta lucrare am urmarit doua criterii importante ale etapelor de dezvoltare si utilizare a algoritmilor inteligenti pe dispozitive mobile. Principalele cazuri studiate in aceasta lucrare sunt: flexibilitatea in construirea unui sistem inteligent si utilizarea operatiilor de inferenta intr-o maniera optima si potrivita fortei de procesare a dispozitivelor mobile.\newline

Pentru a obtine un model cu acuratete optima problemei alese, este nevoie de un proces de testare de configurari si analizare. Acest procedeu este costisitor din punct de vedere al timpului si necesita o flexibilitate la nivel de configurare a parametrilor si a arhitecturii.\newline

Prin accelerarea harware obtinem doar un succes de moment, ce va urma a fi diminuat pe parcurs ce complexitatea aplicatiilor creste. Pentru a obtine un progres consistent si de viitor, este nevoie de dezvoltarea pe diferite planuri concomitent. O constrangere importanta a utilizarii operatiilor invatarii automate pe dispozitive mobile, o reprezinta incapabilitatea dispozitivelor de a utiliza operatii complexe pe un numar mare de fire de executie, fiind limitate de consumul relativ restrans de energie impus de bateriile curente. Pentru a trece peste acest impas, este nevoie de dezvoltarea aplicatiilor tinand cont de arhitecturii de tip RISC si de algoritmi optimizati pentru a reduce complexitatea operatiilor.

