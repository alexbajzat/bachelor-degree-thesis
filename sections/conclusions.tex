\chapter{Concluzii}

Domeniul procesarii de imagini este unul a cărui desfășurare este în amploare si care va fi nelipsit din majoritatea aplicațiilor. Datorită avansării rapide a tehnologiei și a dependenșelor costisitoare ce iși pun amprenta asupra algoritmilor de învățare automată, domeniul va trebui să țină pasul atât din punctul de vedere tehnic cât și din punctul de vedere al algoritmilor și a optimizării. \newline

În aceasta lucrare am urmărit doua criterii importante ale etapelor de dezvoltare și utilizare a algoritmilor inteligenți pe dispozitive mobile. Principalele cazuri studiate în aceasta lucrare sunt: flexibilitatea în construirea unui sistem inteligent și utilizarea operațiilor de inferențî într-o manieră optimă și potrivită forței de procesare a dispozitivelor mobile.\newline

Pentru a obține un model cu acuratețe optimă problemei alese, este nevoie de un proces de testare de configurări si analizare. Acest procedeu este costisitor din punct de vedere al timpului și necesită o flexibilitate la nivel de configurare a parametrilor și a arhitecturii.\newline

Prin accelerarea hardware obținem doar un succes de moment, ce va urma a fi diminuat pe parcurs ce complexitatea aplicațiilor crește. Pentru a obține un progres consistent și de viitor, este nevoie de dezvoltarea pe diferite planuri concomitent. O constrângere importantă a utilizării operațiilor învățării automate pe dispozitive mobile, o reprezintă incapabilitatea dispozitivelor de a utiliza operații complexe pe un număr mare de fire de execuție, fiind limitate de consumul relativ restrâns de energie impus de bateriile curente. Pentru a trece peste acest impas, este nevoie de dezvoltarea aplicatiilor tinând cont de arhitecturii de tip RISC și de algoritmi optimizați pentru a reduce complexitatea operațiilor.

